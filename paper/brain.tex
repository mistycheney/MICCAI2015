\documentclass{llncs}

\usepackage{makeidx}  % allows for indexgeneration

\usepackage{graphics, graphicx, xcolor}


\newcommand{\authcmt}[2]{\textcolor{#1}{#2}}
\newcommand{\yuncong}[1]{\authcmt{red}{[YC: #1]}}
\newcommand{\yoav}[1]{\authcmt{blue}{[YF: #1]}}

\begin{document}
%
%\mainmatter              % start of the contributions
%
\title{Robust Landmark Detection for Mouse Brain Section Images}
%
\titlerunning{Robust Region Landmark Detection}  % abbreviated title (for running head)
%                                     also used for the TOC unless
%                                     \toctitle is used
%
\author{Yuncong Chen \and Yoav Freund}
%
\authorrunning{Yuncong Chen et al.} % abbreviated author list (for running head)
%
%%%% list of authors for the TOC (use if author list has to be modified)
\tocauthor{Yuncong Chen, Yoav Freund}
%
\institute{Department of Computer Science and Engineering, \\University of California, San Diego, La Jolla, CA 92122, USA}




\maketitle              % typeset the title of the contribution

\begin{abstract}

\yoav{Abstract is high priority because deadline is soon}

\keywords{landmark detection, atlas generation, mouse brain, gabor filter}
\end{abstract}
%

\yoav{Overall, good outline. Next steps are to write an abstract and
  to generate good figures. I would write the text after having the
  figures.}

\section{Introduction}
%

Registering brainstem is hard due to the lack of high contrast edges,
compared to Cerebral Cortex and Cerebellum. One has to rely on
differences in texture.

\yoav{The Allen Reference Atlas is only useful to humans, it is not a
  system that can take a stack and align it to a standard. In
  addition, it has limited details in the brain stem. Partha Mitra
  actually had an interesting suggestion, which is that we get the
  atlas images and annotation and use them to train texture
  detectors.}

Allen Reference Atlas does not have enough details in brainstem.

\section{Related Work}

\begin{description}

\item{Point Landmark Detection}

SIFT

\item{Saliency and Objectness Detection}

global rarity scheme

center-surround scheme

\yoav{Is there work that uses notion of statistical significance in
  this context?}

\item{Texture Representation}

gabor filter

textons

\end{description}

\section{Representing Texture using Histograms of Gabor Textons}

Represent texture at each pixel using Gabor filters.

rotation-invariant k-means clustering to form textons.

Over-segment into superpixels.

Describe texture using histogram of textons

\section{Detecting Significant Region Using Center-Surround Contrast}

\yoav{Define the problem of finding regions+textures of high
  statistical significance.}

Grow each superpixel into a region.

Score is computed as the center-surround contrast of the texton histograms.

\section{Human Supervision}

\yoav{Describe how salient region detection improves efficiency of human labeling.}

\section{Detecting Boundaries by Region Concensus}

Instead of voting for each superpixel as boundary separately, vote for them as a set.

Each region votes according to their saliency scores.

Each boundary is described by a tuple that consists of four elements:

1. x-y positions of every superpixel on the boundary

2. centroid of the expansion cluster inside the boundary

3. the average texton distribution of interior superpixels

4. a list of texton distributions of exterior superpixels (more precisely, the closest layer of superpixels on the outside of the boundary).

\section{Matching Boundaries from Different Sections}

distance = interior + exterior + shape + location 
\\

Distance between boundaries are defined as a weighted combination of:

1. Jenson-Shannon divergence between interior distributions

2. symmetric Hausdorff distances between the two sets of exterior distributions. That is, the maximum among the distances between each distribution and its closest distribution from the other set. Here the distance is the Jenson-Shannon divergence.

3. shape dissimilarity: total chi2-distances of shape context descriptors after correspondences are identified for superpixels on two boundaries using dynamic programming. (This is essentially the Shape Distance in Section 5.1 of Belongie's paper)

4. spatial distance: Euclidean distance between cluster centroids
\\

Boundaries are detected from two sections and their pairwise distances are computed. A pair of boundaries are matched if they are the closest boundary of each other.

\section{Experiments}

\subsection{comparison with human labelings}

Shows the results of our algorithm is comparable to human labeling.

show results for RS141. (Do we need to do more than one stacks here? I think one stack already shows enough variations. A coronal stack would be good, but we don't have any now).

\subsection{robustness of matching}

Shows that matchings are robust to distortion and shape change.
Also shows that our distance measure is a sensible one: each of the four terms is important. We show this by changing the term weightings, and then compare matching results.

\section{Future Work}

use detected landmarks for registration

Learn dictionary using deep neural networks, ICA, ...



%
% ---- Bibliography ----
%
\begin{thebibliography}{5}
%
\bibitem {clar:eke}
Clarke, F., Ekeland, I.:
Nonlinear oscillations and
boundary-value problems for Hamiltonian systems.
Arch. Rat. Mech. Anal. 78, 315--333 (1982)

\end{thebibliography}


\end{document}
