\documentclass{llncs}

\usepackage{makeidx}  % allows for indexgeneration

\begin{document}
%
%\mainmatter              % start of the contributions
%
\title{Robust Region Landmark Detection for Mouse Brainstem Section Images}
%
\titlerunning{Robust Region Landmark Detection}  % abbreviated title (for running head)
%                                     also used for the TOC unless
%                                     \toctitle is used
%
\author{Yuncong Chen \and Yoav Freund}
%
\authorrunning{Yuncong Chen et al.} % abbreviated author list (for running head)
%
%%%% list of authors for the TOC (use if author list has to be modified)
\tocauthor{Yuncong Chen, Yoav Freund}
%
\institute{Department of Computer Science and Engineering, \\University of California, San Diego, La Jolla, CA 92122, USA}

\maketitle              % typeset the title of the contribution

\begin{abstract}

\keywords{landmark detection, atlas generation, mouse brain, gabor filter}
\end{abstract}
%
\section{Introduction}
%

Registering brainstem is hard due to the lack of sharp edges, compared to Cerebral Cortex and Cerebellum.

Allen Reference Atlas does not have enough details in brainstem.



\begin{figure}
\vspace{2.5cm}
\caption{}
\end{figure}


\section{Related Work}

\begin{description}

\item{Point Landmark Detection}

SIFT

\item{Saliency and Objectness Detection}

global rarity scheme

center-surround scheme

\item{Texture Representation}

gabor filter

textons

\end{description}

\section{Represent Texture using Histograms of Gabor Textons}

Represent texture at each pixel using Gabor filters.

rotation-invariant k-means clustering to form textons.

Over-segment into superpixels.

Describe texture using histogram of textons


\section{Detect Significant Region Using Center-Surround Contrast}

Region Growing


\section{Robust Boundary Detection by Region Concensus}


\section{Matching Boundaries from Different Sections}



\section{Experiments}

\subsection{comparison with human labelings}

Shows the results of our algorithm is comparable to human labeling.

\begin{figure}
\vspace{2.5cm}
\caption{}
\end{figure}


\subsection{robustness of matching}

Shows that matchings are robust to distortion and shape change.
Also shows that our distance measure is a sensible one: each of the four terms is important. We show this by changing the term weightings, and then compare matching results.



%
% ---- Bibliography ----
%
\begin{thebibliography}{5}
%
\bibitem {clar:eke}
Clarke, F., Ekeland, I.:
Nonlinear oscillations and
boundary-value problems for Hamiltonian systems.
Arch. Rat. Mech. Anal. 78, 315--333 (1982)

\end{thebibliography}


\end{document}
